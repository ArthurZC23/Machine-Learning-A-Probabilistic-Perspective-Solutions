\documentclass[a4paper,10pt]{book}
\usepackage[utf8]{inputenc}
\usepackage{amsmath}

%TODO: finish formatting and read it again

\begin{document}

%Number of the question
\textbf{10.}

%Intro
\textbf{Intro} \par
In this question, we have to derive the inverse gamma density. We will follow the author's suggestion
and use the change of variables formula.\par

%%%%%%%%%%%%%%%%%%%%%%%%%%%%%%%%%%%%%%%%%%%%%%%%%%%%%%%%%%%%%%%%%%%%%%%%%%%%%%%%%%%%%%

%Solution
\textbf{Solution} \par

First, express X as a function of Y.

%Equation 1
\begin{equation}
\begin{split}
X = \frac{1}{Y}
\end{split}
\end{equation}

Second, substitute x by y in the pdf of X:

%Equation 2
\begin{equation}
\begin{split}
p_x(x) = \frac{b^a}{\Gamma(a)}x^{a-1}e^{-bx} = \frac{b^a}{\Gamma(a)}y^{-a+1}e^{-b/y}
\end{split}
\end{equation}

Third, calculat the derivative $\frac{dx}{dy}$.

%Equation 3
\begin{equation}
\begin{split}
\frac{dx}{dy} = -\frac{1}{y^2}
\end{split}
\end{equation}

Now, we just need so substitute (2) and (3) in the expression of change of variables.

%Equation 4
\begin{equation}
\begin{split}
IG(y|a,b) = p_y(y) = p_x(x)|\frac{dx}{dy}| = \frac{b^a}{\Gamma(a)}y^{-a+1}e^{-b/y}\frac{1}{y^2} = \\
\frac{b^a}{\Gamma(a)}y^{-(a+1)}e^{-b/y}
\end{split}
\end{equation}




%Conclusion
\textbf{Conclusion} \par

The main conclusion of this exercise is to see how useful the change of variables formula can be.
However, as stated in this chapter on section 2.7, most of the time, infering a pdf by change of 
variables is difficult. 
Thus, we must always be prepared to use Monte Carlo approximations.

\end{document}


